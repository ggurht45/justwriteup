\chapter{Conclusion}

\label{Chapter9_conclusion} 

\begin{comment}
-------------------------------------------------
%								Chapter layout
9. Conclusion
	a. Future Work
		i. Scoring Functions Improvement
		ii. Other Improvements (UI, distribution, change of framework. electron ui -> web based. )
	b. Personal Note
-------------------------------------------------
\end{comment}
The final version of the application was packaged and distributed to the clinicians. They were happy with final product and have begun initial use of it. This project completed the developmental goals and other functional requirements set out for it and was a successful product. Its algorithms behave well for the a variety of gestures and have the potential to be even extended to other gesture types as they are generic algorithms not tied down to just the 20 gestures considered for this project. This application can be a tool in the arsenal of clinicians who are seeing patients with apraxia and provide an objective measurement of symptoms of that condition. Other than the Leap Motion controller device, the application does not require any additional hardware or setup rig. 


%------------------------------------------------
%	SECTION 1 Future Work
%------------------------------------------------
\section{Future Work}
This application can be extended and made better in several way which can be considered in some future work in this subject area. As was suggested by one of the participants who tested the application in its final form, there could be a game component of the application that could be used by patients who need to practice difficult gestures. In fact, inspiration could be taken from popular games such as Minecraft and an environment could be set up where the user is able to build the virtual worlds using purely hand gestures. This kind of feature would make the application very engaging for the user and also make their time spent practicing gesture very enjoyable and rewarding. 

The gesture recognition algorithms could be improved in some ways also. For one thing, the algorithms could be improved to handle fail cases better. When a human clinician is seeing patients perform gestures, he or she is quickly able to see that if one finger is completely missing the gesture, then the user is not successful in completing the gesture. However, the algorithms currently take an holistic approach to grading and don't have logic that would cause them to return a low score based upon the failure of one or two fingers. Machine learning could also be considered in designing the algorithms. This could allow them to learn the certain parameters that are needed for different patients with different levels of apraxia. 


%------------------------------------------------
%	SECTION 2 Personal Note
%------------------------------------------------
\section{Personal Note}
This project has been quite a journey for me. It has been very exciting and fun but also very challenging at times. It has been super stressful at times and exhilarating at other times. The bugs I was able to find and fix, the late nights spent typing and typing my report. It's been an eventful journey. I have grown a lot as a student and as a person through this project. It has taught me lots of academic knowledge, familiarized me very various technologies that I did not even know before starting this project, and it has taught me about perseverance through difficult times. It is a journey that I will remember as I leave Oxford in the coming weeks. 

For this project, several of the modules I took over the course of this year helped me. Functional programming was very useful as it provided me with a paradigm for structuring and writing cleaner more modular code. Java 1.8 has lambdas which I found to be succinct tools for writing my code especially when lots of inner classes were needed to handle events. Also my module on Visual Analytics helped me to design the UI of my application better. It made me think about which UI features would be the most useful for patients and for clinicians. Concurrency was a big component of my project. Understanding synchronization and multi-threading concepts was a necessary component of developing some of the features that are shown in this project. Therefore, the Concurrency module I took in Hillary Term had impact on my understanding these concepts better and applying them in a real world application. Lastly I would like to mention that the practical experience I gained doing my final project in Requirements Engineering course really helped me to gain confidence in tackling the somewhat intimidating phase of data collection for this project. 

This project also taught me a lot about software engineering as a discipline. I really felt the Agile Development framework is very useful in developing an application quickly. I personally used mini goals every day and every working session to keep myself motivated. These goals would often be broken down into smaller and smaller pieces. This process sounds pretty simple, but it was a crucial component of my work cycle. I also learned to use tools such as Github and Latex more effectively. 

I feel very lucky to have been able to work on such an interesting topic that enabled me to learn and use a host of cool technologies together. It has been a great and memorable experience. 