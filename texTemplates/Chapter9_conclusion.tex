\chapter{Conclusion}

\label{Chapter9_conclusion} 

\begin{comment}
-------------------------------------------------
%								Chapter layout
9. Conclusion
	a. Future Work
		i. Scoring Functions Improvement
		ii. Other Improvements (UI, distribution, change of framework. electron ui -> web based. )
	b. Personal Note
-------------------------------------------------
\end{comment}

This is the conclusion ...

%------------------------------------------------
%	SECTION 1 Future Work
%------------------------------------------------
\section{Future Work}
\begin{comment}
% should talk about how it can be made better. maybe can weigh different fingers differently. maybe in certain gestures a certain finger is more imprortant.

	-angle, dont die if left/right. 
	-need to have a way of returning 0 if one of the fingers is compeletley off. 
	-need to be able to parameterize by finger level. (compare2). if in gesture10 we expect the pinky to be a little curved, should be able to set that parameter on that pinky in that gesture. rather than entire algorithm. need more fine tune parameterization. 
	
	
	
show the 2 figures about loading/creating user. explain why you needed ot make users. meeting with clinicians. working product demonstration. agile. as opposed ot waterfall (this should go in conclusion). 
- in the same thread, could also explain how the rotation of the hand was also the clinician's idea. 
\end{comment}




%------------------------------------------------
%	SECTION 2 Personal Note
%------------------------------------------------
\section{Personal Note}
\begin{comment}

using goals is very helpful. 
break down of tasks. 
also stopping middle of doing something fun. this will make starting again easier. this also helps a lot in writing.
--writing is a big monster.

functional programming is so much better. 

learning things on the run. stackoverflow. learning to debug.

github. branches. going back in history. 
 
latex. 

which course work was useful for the project. functional programming, concurrency, and requirements engineering.










\end{comment}