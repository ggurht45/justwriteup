%----------------------------------------------------------------------------------------
%	PACKAGES AND OTHER DOCUMENT CONFIGURATIONS
%----------------------------------------------------------------------------------------

\documentclass[
11pt, % The default document font size, options: 10pt, 11pt, 12pt
%oneside, % Two side (alternating margins) for binding by default, uncomment to switch to one side
english, % ngerman for German
singlespacing, % Single line spacing, alternatives: onehalfspacing or doublespacing
%draft, % Uncomment to enable draft mode (no pictures, no links, overfull hboxes indicated)
%nolistspacing, % If the document is onehalfspacing or doublespacing, uncomment this to set spacing in lists to single
%liststotoc, % Uncomment to add the list of figures/tables/etc to the table of contents
%toctotoc, % Uncomment to add the main table of contents to the table of contents
%parskip, % Uncomment to add space between paragraphs
%nohyperref, % Uncomment to not load the hyperref package
headsepline, % Uncomment to get a line under the header
%chapterinoneline, % Uncomment to place the chapter title next to the number on one line
%consistentlayout, % Uncomment to change the layout of the declaration, abstract and acknowledgements pages to match the default layout
]{MastersDoctoralThesis} % The class file specifying the document structure

\usepackage[utf8]{inputenc} % Required for inputting international characters
\usepackage[T1]{fontenc} % Output font encoding for international characters

\usepackage{palatino} % Use the Palatino font by default

\usepackage[backend=bibtex,style=trad-plain,natbib=true]{biblatex} % Use the bibtex backend with the authoryear citation style (which resembles APA)

\addbibresource{bib.bib} % The filename of the bibliography

\usepackage[autostyle=true]{csquotes} % Required to generate language-dependent quotes in the bibliography

%----------------------------------------------------------------------------------------
%	MARGIN SETTINGS
%----------------------------------------------------------------------------------------

\geometry{
	paper=a4paper, % Change to letterpaper for US letter
	inner=2.5cm, % Inner margin
	outer=3.8cm, % Outer margin
	bindingoffset=.5cm, % Binding offset
	top=1.5cm, % Top margin
	bottom=1.5cm, % Bottom margin
	%showframe, % Uncomment to show how the type block is set on the page
}

%----------------------------------------------------------------------------------------
%	THESIS INFORMATION
%----------------------------------------------------------------------------------------

\thesistitle{Hand Gesture Recognition via Leap Motion Sensor} % Your thesis title, this is used in the title and abstract, print it elsewhere with \ttitle
\supervisor{Dr. Irina \textsc{Voiculescu}} % Your supervisor's name, this is used in the title page, print it elsewhere with \supname
\examiner{} % Your examiner's name, this is not currently used anywhere in the template, print it elsewhere with \examname
\degree{Masters in Computer Science} % Your degree name, this is used in the title page and abstract, print it elsewhere with \degreename
\author{Jahangir \textsc{Iqbal}} % Your name, this is used in the title page and abstract, print it elsewhere with \authorname
\addresses{} % Your address, this is not currently used anywhere in the template, print it elsewhere with \addressname

\subject{Computer Science} % Your subject area, this is not currently used anywhere in the template, print it elsewhere with \subjectname
\keywords{} % Keywords for your thesis, this is not currently used anywhere in the template, print it elsewhere with \keywordnames
\university{\href{http://www.ox.ac.uk/} {University of Oxford}} %print it elsewhere with \univname
\department{\href{http://www.cs.ox.ac.uk/} {Department of Computer Science}} % print it elsewhere with \deptname


%\group{\href{http://researchgroup.university.com} {Research Group Name}} % Your research group's name and URL, this is used in the title page, print it elsewhere with \groupname
\faculty{\href{http://faculty.university.com}{Faculty Name}} % Your faculty's name and URL, this is used in the title page and abstract, print it elsewhere with \facname

\AtBeginDocument{
\hypersetup{pdftitle=\ttitle} % Set the PDF's title to your title
\hypersetup{pdfauthor=\authorname} % Set the PDF's author to your name
\hypersetup{pdfkeywords=\keywordnames} % Set the PDF's keywords to your keywords
}

%----------------------------------------------------------------------------------------
%	MY CHANGES. START
%----------------------------------------------------------------------------------------

%------------------------------------CONFIGURATION FOR DISPLAYING CODE
\usepackage{listings}
\usepackage{color}

\definecolor{dkgreen}{rgb}{0,0.6,0}
\definecolor{gray}{rgb}{0.5,0.5,0.5}
\definecolor{mauve}{rgb}{0.58,0,0.82}

\lstset{frame=tb,
  language=Java,
  aboveskip=3mm,
  belowskip=3mm,
  showstringspaces=false,
  columns=flexible,
  basicstyle={\small\ttfamily},
  numbers=left,
  stepnumber=1,
  numberstyle=\tiny\color{gray},
  keywordstyle=\color{blue},
  commentstyle=\color{dkgreen},
  stringstyle=\color{mauve},
  breaklines=true,
  breakatwhitespace=true,
  tabsize=3
}

%------------------------------------MULTILINE COMMENTS
\usepackage{verbatim}


%------------------------------------FRAME AROUND CODE LISTING
\usepackage{listings}
\usepackage[most]{tcolorbox}
\usepackage{inconsolata}

\newtcblisting[auto counter]{codelisting}[2][]{sharp corners, 
    fonttitle=\bfseries, colframe=gray, listing only, 
    listing options={basicstyle=\ttfamily,language=java}, 
    title=Listing \thetcbcounter: #2, #1}

%------------------------------------FIGURES SIDE BY SIDE
\usepackage{lipsum}
\usepackage{mwe}%minimum working example

%------------------------------------FIGURES PLACEMENT
\usepackage{float}% If comment this, figure moves to Page 2



%----------------------------------------------------------------------------------------
%	MY CHANGES. END								START OF DOCUMENT BELOW
%----------------------------------------------------------------------------------------

\begin{document}
\frontmatter % Use roman page numbering style (i, ii, iii, iv...) for the pre-content pages
\pagestyle{plain} % Default to the plain heading style until the thesis style is called for the body content

%----------------------------------------------------------------------------------------
%	TITLE PAGE
%----------------------------------------------------------------------------------------
%\begin{comment}
\begin{titlepage}
\begin{center}

\vspace*{.06\textheight}
{\scshape\LARGE \univname\par}\vspace{1.5cm} % University name

\HRule \\[0.4cm] % Horizontal line
{\huge \bfseries \ttitle\par}\vspace{0.4cm} % Thesis title
\HRule \\[1.5cm] % Horizontal line
 
\begin{minipage}[t]{0.4\textwidth}
\begin{flushleft} \large
\emph{Authors:}\\
\href{http://www.johnnysmith.com}{\authorname} % Author name - remove the \href bracket to remove the link
\end{flushleft}
\end{minipage}
\begin{minipage}[t]{0.4\textwidth}
\begin{flushright} \large
\emph{Supervisor:} \\
\href{http://www.jamessmith.com}{\supname} % Supervisor name - remove the \href bracket to remove the link  
\end{flushright}
\end{minipage}\\[3cm]
 
\vfill

\large \textit{A thesis submitted in fulfillment of the requirements\\ for the degree of \degreename}\\[0.3cm] % University requirement text
\textit{in the}\\[0.4cm]
\deptname\\[2cm] % department name
 
\vfill

{\large \today}\\[4cm] % Date
%\includegraphics[scale=.1]{Figures/universityLogo.jpg} % University/department logo - uncomment to place it  
\vfill
\end{center}
\end{titlepage}
%\end{comment}
%----------------------------------------------------------------------------------------
%	DECLARATION PAGE
%----------------------------------------------------------------------------------------
%\begin{comment}
\begin{declaration}
\addchaptertocentry{\authorshipname} % Add the declaration to the table of contents
\noindent I, \authorname, declare that this thesis titled, \enquote{\ttitle} and the work presented in it are my own. I confirm that:

\begin{itemize} 
\item This work was done wholly or mainly while in candidature for a research degree at this University.
\item Where any part of this thesis has previously been submitted for a degree or any other qualification at this University or any other institution, this has been clearly stated.
\item Where I have consulted the published work of others, this is always clearly attributed.
\item Where I have quoted from the work of others, the source is always given. With the exception of such quotations, this thesis is entirely my own work.
\item I have acknowledged all main sources of help.
\item Where the thesis is based on work done by myself jointly with others, I have made clear exactly what was done by others and what I have contributed myself.\\
\end{itemize}
 
\noindent Signed:\\
\rule[0.5em]{25em}{0.5pt} % This prints a line for the signature
 
\noindent Date:\\
\rule[0.5em]{25em}{0.5pt} % This prints a line to write the date
\end{declaration}

\cleardoublepage
%\end{comment}

%----------------------------------------------------------------------------------------
%	ABSTRACT PAGE
%----------------------------------------------------------------------------------------
%\begin{comment}
\begin{abstract}
\addchaptertocentry{\abstractname}
This project created an application which can be used by clinicians working with patients who might have some forms of apraxia, which is a post-stroke condition that makes it difficult to control some muscle groups. These patients are currently diagnosed by human doctors based upon how well they are able to perform certain gestures. The inability to complete these gestures properly can indicate differing levels of apraxia. Of course, there might be slight level of human subjectivity in the diagnosis of this condition. 

This project uses the Leap Motion controller to build an application that can be used by doctors working with patients with possible apraxia. It aims to provide two algorithms that can help to objectively score the gestures performed by patients. The two comparison functions both are successful in grading the gestures and will be discussed in the course of this paper. The project also focused on building a easy to use interface for the clinicians and patients who will be using this application. It is able to take into account the various rotations a user's hand might be doing and undo them appropriately on the screen to show a clear picture of the with the gesture being performed. 

The result of this project is an application that can be used in the real world by clinicians.
\end{abstract}
%\end{comment}

%----------------------------------------------------------------------------------------
%	ACKNOWLEDGEMENTS
%----------------------------------------------------------------------------------------
%\begin{comment}
\begin{acknowledgements}
\addchaptertocentry{\acknowledgementname} % Add the acknowledgements to the table of contents
There are so many people who have helped me to complete this project. It has been a very challenging task at times that was made more attainable through the help and guidance that they provided for me. 

Firstly, I would like to thank my supervisor Dr. Irina Voiculescu. It has been my very good fortune to have found you as my supervisor. Your welcoming smile and your constant encouragements throughout my project have been more helpful than I think I can express. I always left your office feeling happy and with grand plans. Thank you so much for all of your help on this journey!

I would like to thank Dr. Christopher Butler, Dr. Samrah Ahmed, and Nikolas Drummond whom I had the pleasure of working with at John Radcliffe Hospital. They always were so cheerful and supportive of the work I did. 

Finally, I would like to acknowledge my family. Every weekend when I would chat with my grandma and my parents and siblings, I would receive nothing but love. Your belief in me has allowed me to stay up late nights without feeling lonely. I would like to thank you all deeply for your love and support. 
\end{acknowledgements}
%\end{comment}

%----------------------------------------------------------------------------------------
%	LIST OF CONTENTS/FIGURES/TABLES PAGES
%----------------------------------------------------------------------------------------
%\begin{comment}
\tableofcontents % Prints the main table of contents
\listoffigures % Prints the list of figures
%\end{comment}
%----------------------------------------------------------------------------------------
%	DEDICATION
%----------------------------------------------------------------------------------------
%\begin{comment}
\dedicatory{Dedicated to my grandma and my parents.} 
%\end{comment}
%----------------------------------------------------------------------------------------
%	THESIS CONTENT - CHAPTERS
%----------------------------------------------------------------------------------------
\mainmatter % Begin numeric (1,2,3...) page numbering
\pagestyle{thesis} % Return the page headers back to the "thesis" style

% Include the chapters of the thesis as separate files from the Chapters folder
% Uncomment the lines as you write the chapters
%\include{Chapters/Chapter0}
%\include{Chapters/Chapter7_dataCollection}

\chapter{Introduction}

\label{Chapter1_introduction} 

\begin{comment}
-------------------------------------------------
%								Chapter layout
1. Introduction
	a. Motivation
	b. Goals 
	c. Static Hand Gestures 
-------------------------------------------------
\end{comment}
%Computer science as a field is growing rapidly and expanding into many other disciplines including health care. This project is about very 
%short one paragraph or show explanation. maybe write abstract first. 

%------------------------------------------------
%	SECTION 1 Motivation
%------------------------------------------------
\section{Motivation}
Apraxia comes from the prefix “a” meaning “without” and the  Greek root word “praxis” which means “action”. Apraxia is a neurological condition that is sometimes caused by the effects of a stroke. It is a condition in which the patient is unable to exercise motor control over some of their muscles. The muscles themselves are not paralyzed or damaged, what causes these disorders are neurological conditions. In the department of Experimental Psychology researchers are studying patients with post-stroke loss of motor skills including possible cases of apraxia. The diagnosis of such a condition involves researchers presenting their patients with different movement based tests and evaluating the ease and effectiveness with which the patients are able to complete these screenings.  
	
%------------------------------------------------
%	SECTION 2 Goals
%------------------------------------------------
\section{Goals}
%what were the goals of this project and application. 

provide a application that clinicians can use to collect hand gesture data from patients. 
this data should be able to be scored by algorithms to determine how close their attempts are.
there will be two main functions of comparison that will be developed. 
the data collected should be able to be viewed inside the application
the data should be editable from within the app.
stored and retrieved from. 
the data should be saved to csv and outside data should be able to be loaded into the app by csv(as long as hand data is there)
the application should be easy for the users. it should not require a set position of their hand or wrist. 
the gestures shown should be rotateable. to allow from viewing from different angles. 



%------------------------------------------------
%	SECTION 3 Static Hand Gestures 
%------------------------------------------------
\section{Static Hand Gestures}
The gestures this project will be concerned with are ten gestures for the left hand and ten gestures for the right hand. These specific gestures were provided by clinicians during the first meeting with them to discuss the scope of the project. Below are some pictures showing what the target gestures that the users will have to imitate will look like in application. The gestures shown are are the ten gestures for the left hand, along with a rotated version of the same gestures to provide a different perspective into the shape of the gesture. The right hand gestures will be mirror images of the ones shown below. 
%Gesture1Left
\begin{figure}[H]
    \centering
    \begin{minipage}{0.5\textwidth}
        \centering
        \includegraphics[scale=.75]{Figures/gesture1Left.JPG} 
        \caption[Gesture1Left]{Gesture1Left}
		\label{fig:Gesture1Left}
    \end{minipage}\hfill
    \begin{minipage}{0.5\textwidth}
        \centering
        \includegraphics[scale=.7]{Figures/gesture1Left_rotated.JPG}
        \caption[Gesture1Left Rotated]{Gesture1Left rotated.}
        \label{fig:Gesture1Left_rotated}
    \end{minipage}
\end{figure}

%Gesture2Left
\begin{figure}[H]
    \centering
    \begin{minipage}{0.5\textwidth}
        \centering
        \includegraphics[scale=.75]{Figures/gesture2Left.JPG} 
        \caption[Gesture2Left]{Gesture2Left}
		\label{fig:Gesture2Left}
    \end{minipage}\hfill
    \begin{minipage}{0.5\textwidth}
        \centering
        \includegraphics[scale=.75]{Figures/gesture2Left_rotated.JPG}
        \caption[Gesture2Left Rotated]{Gesture2Left rotated.}
        \label{fig:Gesture2Left_rotated}
    \end{minipage}
\end{figure}

%Gesture3Left
\begin{figure}[H]
    \centering
    \begin{minipage}{0.5\textwidth}
        \centering
        \includegraphics[scale=.75]{Figures/gesture3Left.JPG} 
        \caption[Gesture3Left]{Gesture3Left}
		\label{fig:Gesture3Left}
    \end{minipage}\hfill
    \begin{minipage}{0.5\textwidth}
        \centering
        \includegraphics[scale=.75]{Figures/gesture3Left_rotated.JPG}
        \caption[Gesture3Left Rotated]{Gesture3Left rotated.}
        \label{fig:Gesture3Left_rotated}
    \end{minipage}
\end{figure}

%Gesture4Left
\begin{figure}[H]
    \centering
    \begin{minipage}{0.5\textwidth}
        \centering
        \includegraphics[scale=.75]{Figures/gesture4Left.JPG} 
        \caption[Gesture4Left]{Gesture4Left}
		\label{fig:Gesture4Left}
    \end{minipage}\hfill
    \begin{minipage}{0.5\textwidth}
        \centering
        \includegraphics[scale=.75]{Figures/gesture4Left_rotated.JPG}
        \caption[Gesture4Left Rotated]{Gesture4Left rotated.}
        \label{fig:Gesture4Left_rotated}
    \end{minipage}
\end{figure}

%Gesture5Left
\begin{figure}[H]
    \centering
    \begin{minipage}{0.5\textwidth}
        \centering
        \includegraphics[scale=.75]{Figures/gesture5Left.JPG} 
        \caption[Gesture5Left]{Gesture5Left}
		\label{fig:Gesture5Left}
    \end{minipage}\hfill
    \begin{minipage}{0.5\textwidth}
        \centering
        \includegraphics[scale=.75]{Figures/gesture5Left_rotated.JPG}
        \caption[Gesture5Left Rotated]{Gesture5Left rotated.}
        \label{fig:Gesture5Left_rotated}
    \end{minipage}
\end{figure}

%Gesture6Left
\begin{figure}[H]
    \centering
    \begin{minipage}{0.5\textwidth}
        \centering
        \includegraphics[scale=.75]{Figures/gesture6Left.JPG} 
        \caption[Gesture6Left]{Gesture6Left}
		\label{fig:Gesture6Left}
    \end{minipage}\hfill
    \begin{minipage}{0.5\textwidth}
        \centering
        \includegraphics[scale=.7]{Figures/gesture6Left_rotated.JPG}
        \caption[Gesture6Left Rotated]{Gesture6Left rotated.}
        \label{fig:Gesture6Left_rotated}
    \end{minipage}
\end{figure}

%Gesture7Left
\begin{figure}[H]
    \centering
    \begin{minipage}{0.5\textwidth}
        \centering
        \includegraphics[scale=.75]{Figures/gesture7Left.JPG} 
        \caption[Gesture7Left]{Gesture7Left}
		\label{fig:Gesture7Left}
    \end{minipage}\hfill
    \begin{minipage}{0.5\textwidth}
        \centering
        \includegraphics[scale=.75]{Figures/gesture7Left_rotated.JPG}
        \caption[Gesture7Left Rotated]{Gesture7Left rotated.}
        \label{fig:Gesture7Left_rotated}
    \end{minipage}
\end{figure}

%Gesture8Left
\begin{figure}[H]
    \centering
    \begin{minipage}{0.5\textwidth}
        \centering
        \includegraphics[scale=.75]{Figures/gesture8Left.JPG} 
        \caption[Gesture8Left]{Gesture8Left}
		\label{fig:Gesture8Left}
    \end{minipage}\hfill
    \begin{minipage}{0.5\textwidth}
        \centering
        \includegraphics[scale=.75]{Figures/gesture8Left_rotated.JPG}
        \caption[Gesture8Left Rotated]{Gesture8Left rotated.}
        \label{fig:Gesture8Left_rotated}
    \end{minipage}
\end{figure}

%Gesture9Left
\begin{figure}[H]
    \centering
    \begin{minipage}{0.5\textwidth}
        \centering
        \includegraphics[scale=.75]{Figures/gesture9Left.JPG} 
        \caption[Gesture9Left]{Gesture9Left}
		\label{fig:Gesture9Left}
    \end{minipage}\hfill
    \begin{minipage}{0.5\textwidth}
        \centering
        \includegraphics[scale=.75]{Figures/gesture9Left_rotated.JPG}
        \caption[Gesture9Left Rotated]{Gesture9Left rotated.}
        \label{fig:Gesture9Left_rotated}
    \end{minipage}
\end{figure}

%Gesture10Left
\begin{figure}[H]
    \centering
    \begin{minipage}{0.5\textwidth}
        \centering
        \includegraphics[scale=.75]{Figures/gesture10Left.JPG} 
        \caption[Gesture10Left]{Gesture10Left}
		\label{fig:Gesture10Left}
    \end{minipage}\hfill
    \begin{minipage}{0.5\textwidth}
        \centering
        \includegraphics[scale=.75]{Figures/gesture10Left_rotated.JPG}
        \caption[Gesture10Left Rotated]{Gesture10Left rotated.}
        \label{fig:Gesture10Left_rotated}
    \end{minipage}
\end{figure}

\include{Chapters/Chapter2_background}
\include{Chapters/Chapter3_uiHandModel} 
\chapter{Rotation of the Hand UI Model}

\label{Chapter4_rotation} 

\begin{comment}
-------------------------------------------------
%								Chapter layout
4. Rotation of Hand UI Model
	a. Description of Problem
	a. JavaFx Coordinate System vs Leap Motion Coordinate System
	b. Ineffective Leap Motion Data
		i. Pitch Roll Yaw
		ii. Negative Zeros
	c. Simplified Hand Model
	d. Composite Linear Transformations
	e. Rotational Matrix
-------------------------------------------------
\end{comment}
In this chapter the rotation of the Hand UI model will be discussed. 

%------------------------------------------------
%	SECTION 1 Description of Problem
%------------------------------------------------
\section{Description of Problem}
The UI model of the hand is built from the Leap Motion sensor data as discussed in the chapter. This data represents the hand as it is displayed in reality above the Leap Motion controller device. Originally the representation of the hand was built from this hand without any further modifications. However, to demonstrate why this is sometimes not the ideal situation, see Figure \ref{fig:handFlat}. It shows the hand as it is displayed in reality; it is in parallel plane above the device. 
\begin{figure}[H]
\centering
\includegraphics[scale=0.45]{Figures/4_handFlat.JPG}
\caption[Hand in Flat Orientation]{This shows the user's hand in a flat orientation; as determined by the Leap Motion sensor.}
\label{fig:handFlat}
\end{figure}
It should noted how difficult it is to see the fingers and thumb. They are only barely visible. Therefore the user would be forced to force his or her hand into a vertical postion by straining their wrist just so they can see the gesture they are performing on the screen. Figures \ref{fig:handYaw} and \ref{fig:handRoll} show other orientation the hand can take; these figures show the hand after a yaw rotation and after a roll rotation respectively. Again, note the slight difficulty in figuring out the thumb and fingers positions and orientations when the hand is rolled around the z-axis. The hand that is shown in the yaw rotation in Figure \ref{fig:handYaw} is able to seen a little clearly because the wrist was strained upwards. 
\begin{figure}[H]
    \centering
    \begin{minipage}{0.5\textwidth}
        \centering
        \includegraphics[scale=.55]{Figures/4_handYaw.JPG} 
        \caption[Hand with Yaw Rotation]{The user's hand after the certain yaw rotation around the y-axis.}
		\label{fig:handYaw}
    \end{minipage}\hfill
    \begin{minipage}{0.5\textwidth}
        \centering
        \includegraphics[scale=.55]{Figures/4_handRoll.JPG}
        \caption[Hand with Roll Rotation]{The user's hand after the certain roll rotation around the z-axis.}
        \label{fig:handRoll}
    \end{minipage}
\end{figure}
Finally all of these different orientation can of course overlap with each other to create a complex orientation of the hand as shown in Figure \ref{fig:weirdHandShake}, which shows the user's left hand in a weird handshake sort of position while being rotated to the right and tilted upwards.
\begin{figure}[H]
\centering
\includegraphics[scale=0.45]{Figures/4_handWeirdHandshake.JPG}
\caption[Hand in Weird Handshake Position]{This shows the user's hand in a flat orientation; as determined by the Leap Motion sensor.}
\label{fig:weirdHandShake}
\end{figure}
This chapter will explain how these different possible orientations the user's hand might be in while they are performing gestures shown on the screen are accounted for and "undone". Doing this results in the user's hand to be displayed in a set vertical orientation on the screen despite the different ways the user may have oriented their hand above the device in reality. The final resulting hand for all of the figures seen previously  will be as is shown in Figure \ref{fig:handFinalResult} after the composite rotation have transformed it to a vertical position. 
\begin{figure}[H]
\centering
\includegraphics[scale=0.45]{Figures/4_handFinalFix.JPG}
\caption[Hand Fixed to Vertical Orientation]{This shows what the user's hand will look like after all of the possible rotational transforms have been undone and the hand has been fixed to a vertical orientation.}
\label{fig:weirdHandShake}
\end{figure}
This feature of the application makes it easier for the user to see what his or her hand is doing without requiring them to always contain their hand in a specific orientation. The goal of this application is to measure the accuracy with which a user is able to complete certain gestures, not the orientation of their hand. In fact, the algorithms that are used to grade the correctness of the user's attempted gesture do not take the user's hand orientation into account. They focus on the fingers bones and their relative orientations to each other. At the beginning of this project, one of the ideas discussed was to build some sort of rig which would be used to place the user's hand in a set orientation. However because of what will be explained in this chapter such a rig is no longer necessary.



%------------------------------------------------
%	SECTION 1 JavaFx CS vs Leap Motion CS
%------------------------------------------------

\section{JavaFx Coordinate System vs Leap Motion Coordinate System}
	


%------------------------------------------------
%	SECTION 2 Ineffective Leap Motion Data
%------------------------------------------------
\section{Ineffective Leap Motion Data}


%----------------------------------- Pitch Roll Yaw
\subsection{Pitch Roll Yaw}


%----------------------------------- Negative Zeros
\subsection{Negative Zeros}




%------------------------------------------------
%	SECTION 3 Simplified Hand Model
%------------------------------------------------

\section{Simplified Hand Model}
	
	
%------------------------------------------------
%	SECTION 4 Composite Linear Transformations
%------------------------------------------------

\section{Composite Linear Transformations}





%------------------------------------------------
%	SECTION 5 Rotational Matrix
%------------------------------------------------

\section{Rotational Matrix}






\chapter{Scoring of Gestures}

\label{Chapter5_scoring} 

\begin{comment}
-------------------------------------------------
%								Chapter layout
5. Scoring of Gestures
	a. Angle Based Comparison Function
	b. Component Based Comparison Function
-------------------------------------------------
\end{comment}

%------------------------------------------------
%	SECTION 1 Angle Based Comparison Function
%------------------------------------------------
\section{Angle Based Comparison Function}
%description of algorithm and introduction to compare() function
This is the first way by which a hand gesture is scored. This scoring method compares one hand against another; namely, it is usually comparing the user's hand versus the target hand shown on the screen that the user was trying to imitate. It uses the angles between the three foremost bones, (distal, intermediate, proximal), of the five fingers as the primary means of determining how close a user's hand is to the target hand. It also uses the angle the wrist makes to the arm in its calculations to determine the final score for the hand. Figure \ref{fig:compare1} shows some important parts of the compare() function which scores the angular similarities between two hands. This function first makes sure that the two hands that are being compared are of the same kind; ie the hands must both be left or both must be right, otherwise the result of the compare() function will be 0. It then finds angles between adjacent finger bones in both hands and compares the two angles for the amount of similarity between them. This similarity between the two angles, which is the result of the compareAngles() method call, will be a number between 0 and 1. A weight applied to this similarity measure and then the result added to the summation variable x. This function relies on some weight parameters, see Figure \ref{fig:weights}, that are set higher for the longer bones that are closer to the knuckles. For example the proximal bones have a weight of 4; the intermediate bones have a weight of 2 and the distal bones have a weight of 1. This is because the bigger bones closer to the palm of the hand are a bit more limited in their mobility. Therefore, determining correlation between these corresponding bigger bones such as proximal has a higher influence on the overall value of the compare() function. 
\begin{figure}[H]
\centering
\begin{lstlisting}
public double compare(Hand h1, Hand h2) {
	//check if both hands are of the same type.
	if (h1.isLeft() == h2.isLeft()) {
		double x = 0;
		//wrist
		x += compareAngles(angleWristArm(h1), angleWristArm(h2))* weight_wrist;
		//five fingers "proximal". compareAngles always returns between 0-1
		x += compareAngles(anglePinkyProximal(h1), anglePinkyProximal(h2))*weight_pinky_proximal;
		x += compareAngles(angleRingProximal(h1), angleRingProximal(h2))* weight_ring_proximal;
		x += compareAngles(angleMiddleProximal(h1), angleMiddleProximal(h2))* weight_middle_proximal;
		x += compareAngles(angleIndexProximal(h1), angleIndexProximal(h2))* weight_index_proximal;
		x += compareAngles(angleThumbProximal(h1), angleThumbProximal(h2))* weight_thumb_proximal;
		//five fingers "intermediate"
		x += compareAngles(anglePinkyIntermediate(h1), anglePinkyIntermediate(h2))* weight_pinky_intermediate;
		...
		//five fingers "distal"
		x += compareAngles(anglePinkyDistal(h1), anglePinkyDistal(h2))* weight_pinky_distal;
		...
		x /= totalWeight();
		return x;
	} else{
		//if comparing left hand to right hand (or vice versa), return 0
		return 0; 
	}
}
\end{lstlisting}
\caption[Angular Comparison Function]{This snippet of code shows the main skeleton of the function that determines the similarity between two hands by comparing angles between various bones in the hands.}
\label{fig:compare1}
\end{figure}


\begin{figure}[H]
\centering
\begin{lstlisting}
//weights for various bone types 
static double weight_pinky_proximal = 4;
static double weight_pinky_intermediate = 2;
static double weight_pinky_distal = 1;
...
\end{lstlisting}
\caption[Bone Weights in Angular Comparison Function]{An example of the weights set for different bone types in the pinky finger.}
\label{fig:weights}
\end{figure}


%compareAngles() function
It is also worth looking into how the compareAngles() function is defined as this function determines what percentage of the weights get applied. It takes in two angles as it parameters. These angles are determined via various functions which find angles between consecutive bones of a specific finger. An example of one such function is shown in Figure \ref{fig:angleIndexDistal} which shows how the angle between the distal bone and the intermediate bone in the index finger is determined. The angle returned by these functions will always be less than 180 degrees because of the way the angleTo() function is defined in the Leap Motion API. 
\begin{figure}[H]
\centering
\begin{lstlisting}
private float angleIndexDistal(Hand h) {
	Vector direction1 = h.fingers().get(1).bone(Bone.Type.TYPE_DISTAL).direction();
	Vector direction2 = h.fingers().get(1).bone(Bone.Type.TYPE_INTERMEDIATE).direction();
	float rawAngle = direction1.angleTo(direction2);//always less than 180
	return normalize(rawAngle, h);//flips angle on xAxis if palm facing upwards
}
\end{lstlisting}
\caption[angleIndexDistal() Function]{This function is one example of how the angles between adjacent bones are determined.}
\label{fig:angleIndexDistal}
\end{figure}

The compareAngles() is a mathematical function that determines the similarity between two angles passed into it by using the cosine trigonometry function. It will return an number between 0 and 1. It first finds the difference between the two angels. If the angles are so far apart and the distance is greater than 45 degrees, then the function will return a zero to indicate that there is not any meaningful closeness between the two angles being compared. Figure \ref{fig:compareAngles} shows this function's code. 
\begin{figure}[H]
\centering
\begin{lstlisting}
private double compareAngles(float angle1, float angle2) {
	double differenceBtwAngles = Math.abs(angle1-angle2);
	//tmp can be at most pi/4 = 45
	double tmp = Math.min(differenceBtwAngles, Math.PI/4);
	//if tmp is exactly 45, will return 0. cos(90) = 0.
	return Math.cos(2*tmp);
}
\end{lstlisting}
\caption[compareAngles() Function]{This function determines how similar (or close together) two angles using cosine.}
\label{fig:compareAngles}
\end{figure}



	


%------------------------------------------------
%	SECTION 2 Component Based Comparison Function
%------------------------------------------------
\section{Component Based Comparison Function}
The second way by which a score is assigned to a hand representing an attempted gesture will be discussed in this section. 

%descripton of algorithm
The idea behind this method is to take a given hand and decompose it into smaller component that can be scored individually. Then these components scores will be combined to arrive at the cumulative score for the entire hand. Each finger is seen as a component. After considering all of the gestures being tested in this project, I realized that each finger can be in one of three main kinds of poses. All of the fingers except for the thumb are only seen in some variations of being straight, or being curved. The thumb, however, has its own special kind of pose, which deals with connecting to other fingers. For example in some of the gestures the thumb is touching the pinky; in some gestures it is touching the middle finger. Therefore, the third possible pose is represented by a finger name, such as "pinky" or "index" etc., and it represents the finger the thumb is touching in the gesture being analyzed. The way the algorithm is designed, it makes sense to assign this dynamic third pose to the thumb only. Figure \ref{fig:gestureComponents1} and Figure \ref{fig:gestureComponents2} shows one of the gestures used in this project, namely gesture9Left, to illustrate what is meant by the different poses different components of the hand can take. As we can see, all of the fingers are straight; the ring finger is curved; and the thumb is touching the ring finger. This "pose signature" of this gesture as it is used in code is shown in Figure \ref{fig:gesture9PoseSignature}. The pose signature for a certain gesture is the same regardless of whether left or right hand is being used. That is why the code sample shows two case statements for gesture9Left and gesture9Right. 

\begin{figure}[H]
    \centering
    \begin{minipage}{0.45\textwidth}
        \centering
        \includegraphics[scale=.5]{Figures/straight_curved_thumb1.JPG} % first figure itself
        \caption{Gesture showing different finger poses.}
		\label{fig:gestureComponents1}
    \end{minipage}\hfill
    \begin{minipage}{0.45\textwidth}
        \centering
        \includegraphics[scale=.5]{Figures/straight_curved_thumb2.JPG} % second figure itself
        \caption{Same gesture after a 90 degree rotation.}
        \label{fig:gestureComponents2}
    \end{minipage}
\end{figure}


\begin{figure}[H]
\centering
\begin{lstlisting}
case "gesture9Left":
case "gesture9Right":
	fingerPoseMap.put("index", "straight");
	fingerPoseMap.put("middle", "straight");
	fingerPoseMap.put("ring", "curved");
	fingerPoseMap.put("pinky", "straight");
	fingerPoseMap.put("thumb", "ring");
	break;
\end{lstlisting}
\caption[Finger Pose Mapping]{In the component based scroing, each gesture type gets a certain mapping for the kinds of poses fingers are expected to be in for that gesture.}
\label{fig:gesture9PoseSignature}
\end{figure}


% compare function and its helper functions
The comparison function for the component based scoring of hand gestures is shown in Figure \ref{fig:compare2}. It returns a number 0-100 just like the angle based comparison function to indicate the score for the hand being graded. This function gets the fingers for the hand and goes through and finds the individual grades for each finger. Then it combines the into a cumulative grade by weighing the fingers equally. 
\begin{figure}[H]
\centering
\begin{lstlisting}
public static int compare(Hand h, String gestureType) {
	FingerList fingerList = h.fingers();
	//make sure you have five fingers
	if (fingerList.count() == 5) {
		//calculate grades for each finger
		HashMap<String, Double> grades = getFingersGradedMap(getFingerHashMap(fingerList), getFingerPoseMap(gestureType));
		//grade for whole hand
		double totalGrade = cumulativeGrade(grades);
		//score 0-100
		return (int) (totalGrade * 100.0);
	}
	return -1;
}
\end{lstlisting}
\caption[Component Based Comparison Function]{}
\label{fig:compare2}
\end{figure}

To give a clearer idea about how the fingers actually get graded, the gradeFinger() function is shown in Figure \ref{fig:gradeFinger}. This function relies on three helper functions which calculate the straightness and curvedness of fingers and a function which returns the score for the thumb. 
\begin{figure}[H]
\centering
\begin{lstlisting}
private static double gradeFinger(HashMap<String, Finger> fingerMap, Finger f, String pose) {
	if (pose.equals("straight")) {
		return straightnessOfFinger(f);
	} else if (pose.equals("curved")) {
		return curvednessOfFinger(f);
	}
	//thumb is not touching any finger
	else if (pose.equals("thumb")) {
		return straightnessOfFinger(f);
	}
	//thumb touching other fingers
	else {
		Finger theFingerThumbTouches = fingerMap.get(pose);
		return getThumbScore(f, theFingerThumbTouches);
	}
}
\end{lstlisting}
\caption[gradeFinger() Function]{Given a finger a certain pose, this function returns a grade (0-1) for that finger. It uses helper functions to calculate grades for a finger in one the three main kinds of poses.}
\label{fig:gradeFinger}
\end{figure}		

Two of these helper functions, the straightnessOfFinger() and curvednessOfFinger() are shown in Figure \ref{fig:straightCurvedHelperFunctions}. These functions first find the sum of the angle between consecutive bones in the finger that is being graded. For a perfectly straight finger, the sum of these angles should be around 0 degrees. However, to allow for some leniency in the grading 30 degrees are subtracted from the sum of the angles. This allows for a buffer for the user that we intuitively as humans might guage as being relatively straight. For measuring the curvedness of a finger, the sum of the angles between the bones of the fingers should be as close to 270 as possible. However, again a buffer was provided to allow for not perfectly curled fingers to still be valid enough to return a good score.Of course these parameters can be adjusted if this application was used in the real world. These were what I felt were good parameters when I wrote these grading functions.
\begin{figure}[H]
\centering
\begin{lstlisting}
private static double straightnessOfFinger(Finger f) {
	//best case = 0; worst case is: 90+90+90 = 270.
	double sumOfAngles = getSumOfThreeAnglesBetweenFingerBones(f);
	sumOfAngles = sumOfAngles - 30;//offset by 30 degrees
	double score = sumOfAngles / 270;//closer to 0 means a better score
	score = 1 - score;//conventional scale: 0 = bad, 1 = good.
	return snapScore0to1(score);
}
private static double curvednessOfFinger(Finger f) {
	//best case is: 90+90+90 = 270; adjusted bestcase = 210; worst case = 0;
	double sumOfAngles = getSumOfThreeAnglesBetweenFingerBones(f);
	double score = sumOfAngles / 210;//closer to 1 means a better score
	return snapScore0to1(score);
}
\end{lstlisting}
\caption[straightnessOfFinger() and curvednessOfFinger()]{These helper functions are similar to each other. They are used in grading the four fingers.}
\label{fig:straightCurvedHelperFunctions}
\end{figure}		

The helper function getThumbScore(), shown in Figure \ref{fig:getThumbScore} is the more complicated of the three. The way a score is calculated for a thumb is by finding the distance between the tip bone of the thumb and any of the three outermost bones on the finger the thumb is supposed to be touching. The smallest distance is chosen as the tip of the thumb might be closer to any three of the distal, intermediate or proximal bones. This is because some people rest their thumb on the tip of the distal bone, others rest on top of the distal or the intermediate. This distance is scaled down by the smallest bone length multiplied by a scaling factor. Like the other two functions, straightnessOfFinger() and curvednessOfFinger(), the score that is returned is snapped to be between 0-1. 
\begin{figure}[H]
\centering
\begin{lstlisting}
private static double getThumbScore(Finger thumb, Finger otherFinger) {
	//bones in thumb and finger
	HashMap<String, Bone> thumbMap = getHashMapOfBonesFromFinger(thumb);
	HashMap<String, Bone> fingerMap = getHashMapOfBonesFromFinger(otherFinger);
	//get center point of thumb's tip bone
	Vector thumbTip = thumbMap.get("distal").center();
	//finger bones
	Bone d = fingerMap.get("distal");
	Bone i = fingerMap.get("intermediate");
	Bone p = fingerMap.get("proximal");
	//length of bones
	float smallestBoneLength = (Math.min(Math.min(d.length(), i.length()), p.length()));
	//distances from thumb tip to finger bones
	float d1 = thumbTip.distanceTo(d.center());
	float d2 = thumbTip.distanceTo(i.center());
	float d3 = thumbTip.distanceTo(p.center());
	double minDistance = (double) (Math.min(Math.min(d1, d2), d3));
	//scale and score
	double distanceScaledByBoneLength = minDistance / (smallestBoneLength * 3);
	double score = 1 - distanceScaledByBoneLength;
	return snapScore0to1(score);
}
\end{lstlisting}
\caption[getThumbScore() Helper Function]{This function calculates the grade for the thumb that is supposed to be touching one of the four fingers. It calculates this score by distances rather than using angles.}
\label{fig:getThumbScore}
\end{figure}

%doenst need a target hand to compare against. explain why this is good. 
One final thing to note about this comparison function is that it does not rely on a target hand for the comparison. Instead it relies on set gesture poses that are expected for the different gestures to arrive at its score. Therefore, it is more stable form of comparison. If the target hands are themselves not very good examples of the gestures being displayed, the angle based comparison method's performance could be unnecessarily affected. 
 
\chapter{Application User Interface }

\label{Chapter6_appUI} 

\begin{comment}
-------------------------------------------------
%								Chapter layout
6. Application User Interface 
	a. Main Layout
		i. Creating or Loading User
		ii. Saving User Data
		ii. Visual Rotation of Gesture
		iv. Adding a New Gesture
	d. Tabular Display of Data
	e. Writing and Reading from CSV
	f. Artifacts and Distribution
		i. Leap App Store
		ii. IDE Build Process and Batch Script
-------------------------------------------------
figures needed:

homescreen,
analyze screen.
load folder button clicked. opened dialog. 
save to csv opened dialog
load/create new user screen. 
load/create new user screen. (with dropdown showing)
testing screen
testing screen after rotation done. also showing user hand. 
savingData popup. 
table with editing in progress. 
creating new gesture

\end{comment}


In this chapter some of the useful UI features of the application will be discussed. 

%------------------------------------------------
%	SECTION 0 Main Layout
%------------------------------------------------
\section{Main Layout}

The application has three main scenes. The first one is the home screen which shows buttons to take the user to the other two scenes; "Enter Test Mode" button takes the user to the scene which is used for collecting data, and "Analyze Data" button takes the user to a scene that allows him/her to view the collected data \parencite{theKey}. The home screen also contains two radio buttons to allow the user to select which hand (left or right) he/she will be testing with the gestures. Figure  \ref{fig:homeScreen} shows the layout of the home screen. 
\begin{figure}[H]
\centering
\includegraphics[scale=0.35]{Figures/6_homeScreen.JPG}
\caption[Home Screen Layout]{The scene that the user sees initially when they load the application.}
\label{fig:homeScreen}
\end{figure}


%----------------------------------- Creating or Loading User
\subsection{Creating or Loading User}
The user can click on the "Enter Test Mode" to go to the scene where data will be collected. However, before the user can go to the data collection scene, they must first select a previously saved user or create a new user. All of the hand gesture data collected will be stored in an appropriately named folder for the user. Therefore, when the user clicks on the "Enter Test Mode" the first thing that comes up is a small pop-up screen that asks whether the user would like to create a new user or select an older user; see Figure \ref{fig:createNewUser} and Figure \ref{fig:selectOldUser}. 
\begin{figure}[H]
    \centering
    \begin{minipage}{0.5\textwidth}
        \centering
        \includegraphics[scale=.55]{Figures/6_selectUser.JPG} 
        \caption{Pop-up window showing options to create or select user. }
		\label{fig:createNewUser}
    \end{minipage}\hfill
    \begin{minipage}{0.5\textwidth}
        \centering
        \includegraphics[scale=.55]{Figures/6_selectUserDropdown.JPG}%note png, lower case
        \caption{Pop-up window showing the previously created users.}
        \label{fig:selectOldUser}
    \end{minipage}
\end{figure}
%talk about user object and loading users from file. 
The users shows in the dropdown menu are loaded from a CSV file which is used to record them. Whenever a user is created, a new entry is added to the CSV file for that user. As can be expected these operations are handled by using a convenient User class which encapsulates all the data associated with such an object. 

%----------------------------------- Saving User Data
\subsection{Saving User Data}
After having created or selected a user, the user is taken to the screen where he/she can start to proceed to practicing the ten gestures shown for whichever left/right hand was selected on the home screen; see Figure \ref{fig:dataCollectionScene}. On this data collection screen there are five buttons: the Next and Previous buttons cycle through the gestures, the Save button saves the currently being displayed user hand, the "End Testing" button goes back to the home screen, and the Rotate button which rotates the user and target hand a full 360 degrees slowly.
\begin{figure}[H]
\centering
\includegraphics[scale=0.35]{Figures/6_userTargetHand.JPG}
\caption[Data Collection Scene]{This scene is where the user can complete the shown gestures on the screen and the clinicians can collect the user's gesture data.}
\label{fig:dataCollectionScene}
\end{figure}
There are also some keyboard shortcuts that were coded that function in place of some of the buttons. For example, left and right arrows are mapped to the Previous and Next buttons, while the Enter key is mapped to causing the Save Gesture dialog window to pop up just as the Save button does. The user's hand will appear in full blue color on the screen, whereas the target hand the user is trying to imitate will appear in a dark green mesh material. The user's hand will be saved correctly regardless of whether it is exactly covering the target hand. The target hand is just there to give the user indication of the gesture he/she is doing. Figure \ref{fig:saveGestureDialog} shows the Save Gesture dialog which allows the clinicians to type some comments and mark the result of the user's attempt in replicating the shown gesture. 
\begin{figure}[H]
\centering
\includegraphics[scale=0.35]{Figures/6_saveGestureDialog.JPG}
\caption[Save Gesture Dialog]{This dialog lets the clinicians grade and comment on the user's gesture before saving it.}
\label{fig:saveGestureDialog}
\end{figure}
By default the result is saved to be "Yes" which indicates the user successfully completed the gesture. In the code, there is an object called HandInfo which represents the data being saved including the full file path of where the Leap Motion Hand object will be serialized to, and the comments and the result of the gesture attempt. 

%----------------------------------- Visual Rotation of Gesture
\subsection{Visual Rotation of Gesture}
The Rotate button causes the 3D camera that is being used in the application to spin around. The user and target hands themselves do not move at all, it is the camera that orbits around them while being focused on them. A picture taken while the rotation was happening is shown in Figure \ref{fig:rotationButtonDemo}.
\begin{figure}[H]
\centering
\includegraphics[scale=0.35]{Figures/6_rotationButtonDemo.JPG}
\caption[Rotation Button in Action]{This figure shows instance of the rotation in action. The hands are shown at 90 degree angle because of the camera rotating around them.}
\label{fig:rotationButtonDemo}
\end{figure}

The way this is achieved via code is shown in Figure \ref{fig:visualRotationCode1}. In order for the user to observe the rotation happen in real time on the screen, a Timeline object was used to set up an animation which can update the observable angleProperty() of a rotation transform object  attached to the camera in advance. The timeline was set to last for seven seconds and the starting and ending angles were 0 and 360 degrees respectively. The code which actually initializes and adds the Rotate transform to the 3D perspective camera of the application is shown in Figure \ref{fig:visualRotationCode2}. 
\begin{figure}[H]
\centering
\begin{lstlisting}
//set up rotation timeline
Timeline timeline = new Timeline(
	new KeyFrame(Duration.seconds(0), new KeyValue(rotateAroundY.angleProperty(), 0)),
	new KeyFrame(Duration.seconds(7), new KeyValue(rotateAroundY.angleProperty(), -360)));
timeline.setCycleCount(1);
... 
//rotate button plays animation
rotateButton = new Button("Rotate") {
	@Override
	public void fire() {
	timeline.play();}
};
\end{lstlisting}
\caption[Rotation Animation Timeline]{This code shows the Timeline object that was used to animate the movement of the 3D camera around the y-axis.}
\label{fig:visualRotationCode1}
\end{figure}
Since the "rotateAround" transform object is added first to the camera's transforms, it will be the last transform to be executed. This is exactly what we want so the camera will remain focused on the hands as it rotates. 
\begin{figure}[H]
\centering
\begin{lstlisting}
// The 3D camera
PerspectiveCamera camera = new PerspectiveCamera(true);
// The rotation transform to be updated later
rotateAroundY = new Rotate(0, Rotate.Y_AXIS);
// rotation transform is added first. It will be executed last
camera.getTransforms().addAll(rotateAroundY, new Translate(0, -5, -50), new Rotate(-10, Rotate.X_AXIS));
\end{lstlisting}
\caption[Rotation Transform on Camera]{This code shows the rotate transform that's added to the camera. This transform gets updated by the animation timeline object.}
\label{fig:visualRotationCode2}
\end{figure}

The hands themselves are not affected at all by the rotation and in fact during the seven seconds in which the camera is being rotated around the y-axis, the user can continue trying to imitate the gesture. However, during testing of the application it was found out that most users prefer to have to rotation happen in separately initially so they can just observe. Only after the rotation finishes is when they would start trying to perform the gesture. 
	
%----------------------------------- Adding a New Gesture
\subsection{Adding a New Gesture}
From the data collection scene page, it is also possible to create a new target gesture and add it to the stock of target gestures being considered. This functionality was not really required by the clinicians, however it did come in useful when I was developing the application and it does have the potential of becoming a useful feature if the application is ever deployed into a real life environment. Currently the application interface does not have a button the user or clinicians can press to bring up the Save Gesture dialog window. Instead, there is a keyboard command that is listened for in the code and which is triggered by pressing the letter "G". This brings up the Save Gesture dialog, a picture of which is shown in Figure \ref{fig:saveGesture}. 
\begin{figure}[H]
\centering
\includegraphics[scale=0.45]{Figures/6_createNewGesture.JPG}
\caption[Saving New Gesture]{This dialog window will save the current user hand as a new gesture, in the list of target gestures.}
\label{fig:saveGesture}
\end{figure}
Saving a new gesture will add it to the end of the list of currently used gestures. 

%------------------------------------------------
%	SECTION 3 Tabular Display of Data
%------------------------------------------------
\section{Tabular Display of Data}
If the user clicks on the "Analyze Data" button on the home screen, he/she will be taken to a scene shown in Figure \ref{fig:analyzeData}. This scene shows all of the collected data for the currently selected user in a table on the left. The user is able to select rows in the table by pressing the Up or Down keys on the keyboard or by clicking on a row with their mouse. 
\begin{figure}[H]
\centering
\includegraphics[scale=0.45]{Figures/6_analyzeScreen.JPG}
\caption[Analyze Data Scene]{The scene which allows user to view and edit some of the meta-data on the hand objects saved for the currently loaded user.}
\label{fig:analyzeData}
\end{figure}
This table is sortable by the columns and the columns can also be rearranged to be in a different order. In addition, some of the columns are actually editable and allow the user to specify changes to the data recorded. These changes will be updated for the user when the application is closed or the user navigates to a different scene. The way in which this table is constructed and made to be editablee is shown in the snippets of code in Figure \ref{fig:tabularDataCode}. This code is all from the controller Java class that is set for this scene. A part of this scene was built using Scene Builder and part of it was procedurally created using just Java code. The table section was prototyped in Scene Builder and the section of the scene to the right showing the two hands was created in Scene Builder.
\begin{figure}[H]
\centering
\begin{lstlisting}
//bind references with fxml annotations
@FXML
private TreeTableView<HandInfo2> treeTableView;
@FXML
private TreeTableColumn<HandInfo2, String> col3;
...
//set up results column; using lambda function
col3.setCellValueFactory((TreeTableColumn.CellDataFeatures<HandInfo2, String> param) -> param.getValue().getValue().result2Property());
//set up results column to be editable
ObservableList<String> list = FXCollections.observableArrayList();
list.add("Yes");
list.add("No");
col3.setCellFactory(ChoiceBoxTreeTableCell.forTreeTableColumn(list));
col3.setOnEditCommit(new EventHandler<TreeTableColumn.CellEditEvent<HandInfo2, String>>() {
	@Override
	public void handle(TreeTableColumn.CellEditEvent<HandInfo2, String> event) {
		TreeItem<HandInfo2> h = treeTableView.getTreeItem(event.getTreeTablePosition().getRow());
		h.getValue().setResult(event.getNewValue());
	}
});
...
//make table editable
treeTableView.setEditable(true);
\end{lstlisting}
\caption[TreeTableView Editable Results Column]{These sections of code show how the table and a specific column were accessed from the FXML file. The code also shows how the column was set up to be editable by the user.}
\label{fig:tabularDataCode}
\end{figure}







discuss annotation. reference background. --maybe
discuss lambda function. observable property. finish up. why need to be editable. explain use case as discussed with clinicians. comments section later. file path changes. etc. have some good discussion. 




\begin{comment}

editable picture. show code about how table was constructed. maybe even discuss scenebuilder. 

 The user can click on the "Analyze Data" button to go to the 
talk about three scenes and show pics. show what happens where and 
what the typical user would do. 
take data and analyze it for problems. 
and then go fix those problems. 
also save data to csv. read from csv. 
say some things will be dicussed in other sections. 
\end{comment}





%------------------------------------------------
%	SECTION 4 Writing and Reading from CSV
%------------------------------------------------
\section{Writing and Reading from CSV}


%------------------------------------------------
%	SECTION 5 Artifacts and Distribution
%------------------------------------------------
\section{Artifacts and Distribution}

maybe this section can all be collapsed into one. (IDE Build Process and Batch Script)

%----------------------------------- Leap App Store
\subsection{Leap App Store}

%----------------------------------- IDE Build Process and Batch Script
\subsection{IDE Build Process and Batch Script}



\include{Chapters/Chapter7_dataCollection}
\include{Chapters/Chapter8_conclusion}


%----------------------------------------------------------------------------------------
%	THESIS CONTENT - APPENDICES
%----------------------------------------------------------------------------------------

\appendix % Cue to tell LaTeX that the following "chapters" are Appendices

% Include the appendices of the thesis as separate files from the Appendices folder
% Uncomment the lines as you write the Appendices
\chapter{Participant Survey}

\label{Appendix_survey} 
\begin{verbatim}
Participant Survey

Dear Participant, 
Thank you so much for taking your time to fill out this questionnaire. Your 
answers will constitute as important data for my project concerning Leap 
Motion Gesture Recognition. If you feel uncomfortable about a question, you 
do not have to answer it. 
-------------------------------------------------------------------------------
    1. What is your gender? 
        a. Male 
        b. Female

    2. What is your age?
        a. Teenager
        b. Young Adult 
        c. Middle Age
        d. Elderly

    3. Do you think the vertical readjustment of the user’s hand is helpful?
        a. Yes
        b. No

    4. Any other comments:
	
\end{verbatim}
\chapter{Participant Data}

\label{Appendix_dataAvgs} 
gesture1Left
gesture1Left, dataOutput/General/2017-08-10 01-50-09.hand, c1, Yes, 82, 91
gesture1Left, dataOutput/Alex/2017-08-18 12-54-48.hand, na, Yes, 93, 94
gesture1Left, dataOutput/Alex/2017-08-18 12-58-08.hand, na, Yes, 78, 78
gesture1Left, dataOutput/Jacqueline/2017-08-10 16-07-08.hand, na, Yes, 79, 86
gesture1Left, dataOutput/Stefan/2017-08-23 17-38-35.hand, na, Yes, 76, 75
gesture1Left, dataOutput/test1/2017-08-22 22-31-02.hand, 1, Yes, 77, 92
gesture1Left, dataOutput/test2/2017-08-22 22-45-46.hand, 1, Yes, 96, 94
gesture1Left, dataOutput/test3/2017-08-22 23-28-55.hand, 1, Yes, 84, 77
gesture1Left, dataOutput/test4/2017-08-23 00-00-55.hand, 1, Yes, 95, 91
gesture1Left, dataOutput/test5/2017-08-23 00-26-20.hand, na, Yes, 80, 89
gesture1Left, dataOutput/test6/2017-08-24 02-27-53.hand, na, Yes, 80, 92

gesture2Left
gesture2Left, dataOutput/General/2017-08-10 01-50-31.hand, c2, Yes, 87, 80
gesture2Left, dataOutput/Alex/2017-08-18 12-54-56.hand, na, Yes, 82, 76
gesture2Left, dataOutput/Alex/2017-08-18 12-58-11.hand, na, Yes, 78, 76
gesture2Left, dataOutput/Jacqueline/2017-08-10 16-07-28.hand, na, Yes, 93, 78
gesture2Left, dataOutput/Stefan/2017-08-23 17-38-38.hand, na, Yes, 84, 85
gesture2Left, dataOutput/test1/2017-08-22 22-31-42.hand, 1, Yes, 88, 82
gesture2Left, dataOutput/test2/2017-08-22 22-46-26.hand, 1, Yes, 87, 83
gesture2Left, dataOutput/test3/2017-08-22 23-30-13.hand, 1, Yes, 75, 62
gesture2Left, dataOutput/test4/2017-08-23 00-01-17.hand, na, Yes, 96, 89
gesture2Left, dataOutput/test5/2017-08-23 00-26-35.hand, na, Yes, 85, 82
gesture2Left, dataOutput/test6/2017-08-24 02-28-41.hand, na, Yes, 90, 86

gesture3Left
gesture3Left, dataOutput/General/2017-08-10 01-50-41.hand, c3, Yes, 89, 89
gesture3Left, dataOutput/Alex/2017-08-18 12-55-01.hand, na, Yes, 94, 78
gesture3Left, dataOutput/Alex/2017-08-18 12-58-15.hand, na, Yes, 87, 83
gesture3Left, dataOutput/Jacqueline/2017-08-10 16-08-05.hand, na, Yes, 91, 86
gesture3Left, dataOutput/Stefan/2017-08-23 17-38-42.hand, na, Yes, 84, 85
gesture3Left, dataOutput/test1/2017-08-22 22-32-07.hand, 1, Yes, 93, 87
gesture3Left, dataOutput/test2/2017-08-22 22-46-50.hand, 1, Yes, 85, 75
gesture3Left, dataOutput/test3/2017-08-22 23-30-38.hand, 1, Yes, 85, 89
gesture3Left, dataOutput/test4/2017-08-23 00-02-52.hand, na, Yes, 58, 73
gesture3Left, dataOutput/test5/2017-08-23 00-26-45.hand, na, Yes, 81, 88
gesture3Left, dataOutput/test6/2017-08-24 02-29-02.hand, na, Yes, 75, 83

gesture4Left
gesture4Left, dataOutput/General/2017-08-10 01-50-55.hand, c4, Yes, 86, 88
gesture4Left, dataOutput/Alex/2017-08-18 12-55-06.hand, na, Yes, 96, 92
gesture4Left, dataOutput/Alex/2017-08-18 12-58-17.hand, na, Yes, 96, 97
gesture4Left, dataOutput/Jacqueline/2017-08-10 16-09-39.hand, na, Yes, 94, 96
gesture4Left, dataOutput/Stefan/2017-08-23 17-38-45.hand, na, Yes, 78, 76
gesture4Left, dataOutput/test1/2017-08-22 22-33-12.hand, 1, Yes, 95, 97
gesture4Left, dataOutput/test2/2017-08-22 22-47-12.hand, 1, Yes, 95, 94
gesture4Left, dataOutput/test3/2017-08-22 23-31-12.hand, 1, Yes, 94, 92
gesture4Left, dataOutput/test4/2017-08-23 00-03-46.hand, na, Yes, 71, 80
gesture4Left, dataOutput/test5/2017-08-23 00-26-52.hand, na, Yes, 95, 96
gesture4Left, dataOutput/test6/2017-08-24 02-29-28.hand, na, Yes, 95, 94

gesture5Left
gesture5Left, dataOutput/General/2017-08-10 01-51-08.hand, c5, Yes, 93, 84
gesture5Left, dataOutput/Alex/2017-08-18 12-55-10.hand, na, Yes, 11, 92
gesture5Left, dataOutput/Alex/2017-08-18 12-58-20.hand, na, Yes, 14, 92
gesture5Left, dataOutput/Jacqueline/2017-08-10 16-09-58.hand, na, Yes, 84, 81
gesture5Left, dataOutput/Stefan/2017-08-23 17-38-48.hand, na, Yes, 84, 86
gesture5Left, dataOutput/test1/2017-08-22 22-33-49.hand, 1, Yes, 4, 90
gesture5Left, dataOutput/test2/2017-08-22 22-47-49.hand, 1, Yes, 13, 92
gesture5Left, dataOutput/test3/2017-08-22 23-31-42.hand, 1, Yes, 10, 91
gesture5Left, dataOutput/test4/2017-08-23 00-04-04.hand, na, Yes, 90, 82
gesture5Left, dataOutput/test5/2017-08-23 00-27-36.hand, na, Yes, 83, 86
gesture5Left, dataOutput/test6/2017-08-24 02-30-15.hand, na, Yes, 82, 89

gesture6Left
gesture6Left, dataOutput/General/2017-08-10 01-51-44.hand, c6, Yes, 77, 84
gesture6Left, dataOutput/Alex/2017-08-18 12-55-14.hand, na, Yes, 80, 67
gesture6Left, dataOutput/Alex/2017-08-18 12-58-23.hand, na, Yes, 79, 70
gesture6Left, dataOutput/Jacqueline/2017-08-10 16-10-14.hand, na, Yes, 93, 92
gesture6Left, dataOutput/Stefan/2017-08-23 17-38-51.hand, na, Yes, 88, 79
gesture6Left, dataOutput/test1/2017-08-22 22-34-05.hand, 1, Yes, 77, 73
gesture6Left, dataOutput/test2/2017-08-22 22-48-14.hand, 1, Yes, 93, 85
gesture6Left, dataOutput/test3/2017-08-22 23-32-23.hand, 1, Yes, 88, 76
gesture6Left, dataOutput/test4/2017-08-23 00-04-35.hand, na, Yes, 82, 78
gesture6Left, dataOutput/test5/2017-08-23 00-27-45.hand, na, Yes, 88, 84
gesture6Left, dataOutput/test6/2017-08-24 02-30-40.hand, na, Yes, 98, 92

gesture7Left
gesture7Left, dataOutput/General/2017-08-10 01-51-54.hand, c7, Yes, 74, 55
gesture7Left, dataOutput/Alex/2017-08-18 12-55-20.hand, na, Yes, 2, 89
gesture7Left, dataOutput/Alex/2017-08-18 12-58-27.hand, na, Yes, 4, 81
gesture7Left, dataOutput/Jacqueline/2017-08-10 16-10-38.hand, na, Yes, 88, 85
gesture7Left, dataOutput/Stefan/2017-08-23 17-38-55.hand, na, Yes, 0, 77
gesture7Left, dataOutput/test1/2017-08-22 22-34-28.hand, 1, Yes, 0, 78
gesture7Left, dataOutput/test2/2017-08-22 22-51-11.hand, 1, Yes, 0, 69
gesture7Left, dataOutput/test3/2017-08-22 23-32-46.hand, 1, Yes, 0, 73
gesture7Left, dataOutput/test4/2017-08-23 00-04-57.hand, na, Yes, 81, 91
gesture7Left, dataOutput/test5/2017-08-23 00-28-01.hand, na, Yes, 0, 82
gesture7Left, dataOutput/test6/2017-08-24 02-31-04.hand, na, Yes, 71, 75

gesture8Left
gesture8Left, dataOutput/General/2017-08-10 01-52-05.hand, c8, Yes, 86, 84
gesture8Left, dataOutput/Alex/2017-08-18 12-55-25.hand, na, Yes, 82, 79
gesture8Left, dataOutput/Alex/2017-08-18 12-58-32.hand, na, Yes, 78, 72
gesture8Left, dataOutput/Jacqueline/2017-08-10 16-10-46.hand, na, Yes, 91, 76
gesture8Left, dataOutput/Stefan/2017-08-23 17-38-58.hand, na, Yes, 8, 79
gesture8Left, dataOutput/test1/2017-08-22 22-34-46.hand, 1, Yes, 93, 88
gesture8Left, dataOutput/test2/2017-08-22 22-51-54.hand, 1, Yes, 96, 86
gesture8Left, dataOutput/test3/2017-08-22 23-33-15.hand, 1, Yes, 95, 80
gesture8Left, dataOutput/test4/2017-08-23 00-05-08.hand, na, Yes, 91, 85
gesture8Left, dataOutput/test5/2017-08-23 00-28-09.hand, na, Yes, 91, 88
gesture8Left, dataOutput/test6/2017-08-24 02-31-19.hand, na, Yes, 93, 78

gesture9Left
gesture9Left, dataOutput/General/2017-08-10 01-52-19.hand, c9, Yes, 94, 73
gesture9Left, dataOutput/Alex/2017-08-18 12-55-36.hand, na, Yes, 92, 80
gesture9Left, dataOutput/Alex/2017-08-18 12-58-35.hand, na, Yes, 82, 78
gesture9Left, dataOutput/Jacqueline/2017-08-10 16-11-04.hand, na, Yes, 93, 86
gesture9Left, dataOutput/Stefan/2017-08-23 17-39-03.hand, na, Yes, 86, 80
gesture9Left, dataOutput/test1/2017-08-22 22-35-08.hand, 1, Yes, 85, 87
gesture9Left, dataOutput/test2/2017-08-22 22-52-48.hand, 1, Yes, 94, 87
gesture9Left, dataOutput/test3/2017-08-22 23-33-41.hand, 1, Yes, 94, 80
gesture9Left, dataOutput/test4/2017-08-23 00-06-18.hand, na, Yes, 81, 65
gesture9Left, dataOutput/test5/2017-08-23 00-28-19.hand, na, Yes, 92, 85
gesture9Left, dataOutput/test6/2017-08-24 02-31-39.hand, na, Yes, 92, 90

gesture10Left
gesture10Left, dataOutput/General/2017-08-10 01-52-39.hand, c10, Yes, 88, 70
gesture10Left, dataOutput/Alex/2017-08-18 12-55-46.hand, na, Yes, 73, 67
gesture10Left, dataOutput/Alex/2017-08-18 12-58-39.hand, na, Yes, 80, 71
gesture10Left, dataOutput/Jacqueline/2017-08-10 16-11-14.hand, na, Yes, 93, 80
gesture10Left, dataOutput/Stefan/2017-08-23 17-39-06.hand, na, Yes, 71, 77
gesture10Left, dataOutput/test1/2017-08-22 22-35-43.hand, 1, Yes, 65, 86
gesture10Left, dataOutput/test2/2017-08-22 22-53-14.hand, 1, Yes, 86, 63
gesture10Left, dataOutput/test3/2017-08-22 23-33-56.hand, 1, Yes, 78, 79
gesture10Left, dataOutput/test4/2017-08-23 00-06-36.hand, na, Yes, 70, 75
gesture10Left, dataOutput/test5/2017-08-23 00-28-31.hand, na, Yes, 90, 86
gesture10Left, dataOutput/test6/2017-08-24 02-31-57.hand, na, Yes, 73, 85

gesture1Right
gesture1Right, dataOutput/General/2017-08-10 01-53-51.hand, c1, Yes, 75, 70
gesture1Right, dataOutput/Alex/2017-08-18 12-56-27.hand, na, Yes, 83, 92
gesture1Right, dataOutput/Alex/2017-08-18 12-57-11.hand, na, Yes, 72, 81
gesture1Right, dataOutput/Jacqueline/2017-08-10 16-01-11.hand, na, Yes, 5, 94
gesture1Right, dataOutput/Stefan/2017-08-23 17-37-33.hand, na, Yes, 77, 77
gesture1Right, dataOutput/test1/2017-08-22 22-36-22.hand, 1, Yes, 74, 75
gesture1Right, dataOutput/test2/2017-08-22 22-40-59.hand, 1, Yes, 81, 91
gesture1Right, dataOutput/test3/2017-08-22 23-34-34.hand, 1, Yes, 88, 98
gesture1Right, dataOutput/test4/2017-08-23 00-07-17.hand, na, Yes, 73, 83
gesture1Right, dataOutput/test5/2017-08-23 00-22-17.hand, na, Yes, 88, 98
gesture1Right, dataOutput/test6/2017-08-24 02-32-58.hand, na, Yes, 90, 97

gesture2Right
gesture2Right, dataOutput/General/2017-08-10 01-54-00.hand, c2, Yes, 87, 80
gesture2Right, dataOutput/Alex/2017-08-18 12-56-34.hand, na, Yes, 92, 76
gesture2Right, dataOutput/Alex/2017-08-18 12-57-14.hand, na, Yes, 90, 87
gesture2Right, dataOutput/Jacqueline/2017-08-10 16-02-04.hand, na, Yes, 14, 82
gesture2Right, dataOutput/Stefan/2017-08-23 17-37-38.hand, na, Yes, 84, 70
gesture2Right, dataOutput/test1/2017-08-22 22-36-38.hand, 1, Yes, 87, 81
gesture2Right, dataOutput/test2/2017-08-22 22-41-53.hand, 1, Yes, 5, 72
gesture2Right, dataOutput/test3/2017-08-22 23-34-48.hand, 1, Yes, 82, 72
gesture2Right, dataOutput/test4/2017-08-23 00-08-06.hand, na, Yes, 86, 78
gesture2Right, dataOutput/test5/2017-08-23 00-22-51.hand, na, Yes, 3, 84
gesture2Right, dataOutput/test6/2017-08-24 02-33-07.hand, na, Yes, 14, 80

gesture3Right
gesture3Right, dataOutput/General/2017-08-10 01-54-13.hand, c3, Yes, 96, 76
gesture3Right, dataOutput/Alex/2017-08-18 12-56-39.hand, na, Yes, 84, 88
gesture3Right, dataOutput/Alex/2017-08-18 12-57-22.hand, na, Yes, 78, 80
gesture3Right, dataOutput/Jacqueline/2017-08-10 16-02-35.hand, na, Yes, 82, 87
gesture3Right, dataOutput/Stefan/2017-08-23 17-37-42.hand, na, Yes, 82, 82
gesture3Right, dataOutput/test1/2017-08-22 22-37-06.hand, 1, Yes, 78, 82
gesture3Right, dataOutput/test2/2017-08-22 22-42-48.hand, 1, Yes, 90, 83
gesture3Right, dataOutput/test3/2017-08-22 23-35-01.hand, 1, Yes, 98, 85
gesture3Right, dataOutput/test4/2017-08-23 00-08-36.hand, na, Yes, 95, 79
gesture3Right, dataOutput/test5/2017-08-23 00-23-06.hand, na, Yes, 91, 79
gesture3Right, dataOutput/test6/2017-08-24 02-33-35.hand, na, Yes, 87, 88

gesture4Right
gesture4Right, dataOutput/General/2017-08-10 01-54-22.hand, c4, Yes, 86, 90
gesture4Right, dataOutput/Alex/2017-08-18 12-56-43.hand, na, Yes, 93, 98
gesture4Right, dataOutput/Alex/2017-08-18 12-57-26.hand, na, Yes, 88, 93
gesture4Right, dataOutput/Jacqueline/2017-08-10 16-02-51.hand, na, Yes, 90, 94
gesture4Right, dataOutput/Stefan/2017-08-23 17-37-46.hand, na, Yes, 81, 82
gesture4Right, dataOutput/test1/2017-08-22 22-37-44.hand, 1, Yes, 72, 75
gesture4Right, dataOutput/test2/2017-08-22 22-43-07.hand, 1, Yes, 64, 86
gesture4Right, dataOutput/test3/2017-08-22 23-35-24.hand, 1, Yes, 94, 88
gesture4Right, dataOutput/test4/2017-08-23 00-08-48.hand, na, Yes, 77, 87
gesture4Right, dataOutput/test5/2017-08-23 00-23-36.hand, na, Yes, 0, 88
gesture4Right, dataOutput/test6/2017-08-24 02-34-03.hand, na, Yes, 93, 95

gesture5Right
gesture5Right, dataOutput/General/2017-08-10 01-54-43.hand, c5, Yes, 89, 88
gesture5Right, dataOutput/Alex/2017-08-18 12-56-47.hand, na, Yes, 96, 87
gesture5Right, dataOutput/Alex/2017-08-18 12-57-31.hand, na, Yes, 97, 87
gesture5Right, dataOutput/Jacqueline/2017-08-10 16-03-01.hand, na, Yes, 91, 80
gesture5Right, dataOutput/Stefan/2017-08-23 17-37-49.hand, na, Yes, 94, 91
gesture5Right, dataOutput/test1/2017-08-22 22-37-59.hand, 1, Yes, 88, 80
gesture5Right, dataOutput/test2/2017-08-22 22-43-26.hand, 1, Yes, 85, 82
gesture5Right, dataOutput/test3/2017-08-22 23-35-36.hand, 1, Yes, 96, 86
gesture5Right, dataOutput/test4/2017-08-23 00-09-14.hand, na, Yes, 80, 67
gesture5Right, dataOutput/test5/2017-08-23 00-23-52.hand, na, Yes, 83, 87
gesture5Right, dataOutput/test6/2017-08-24 02-34-14.hand, na, Yes, 87, 62

gesture6Right
gesture6Right, dataOutput/General/2017-08-10 01-55-24.hand, c6, Yes, 89, 89
gesture6Right, dataOutput/Alex/2017-08-18 12-56-51.hand, na, Yes, 5, 77
gesture6Right, dataOutput/Alex/2017-08-18 12-57-34.hand, na, Yes, 78, 85
gesture6Right, dataOutput/Jacqueline/2017-08-10 16-03-12.hand, na, Yes, 73, 78
gesture6Right, dataOutput/Stefan/2017-08-23 17-37-53.hand, na, Yes, 8, 66
gesture6Right, dataOutput/test1/2017-08-22 22-38-10.hand, 1, Yes, 87, 85
gesture6Right, dataOutput/test2/2017-08-22 22-43-43.hand, 1, Yes, 82, 72
gesture6Right, dataOutput/test3/2017-08-22 23-35-57.hand, 1, Yes, 83, 88
gesture6Right, dataOutput/test4/2017-08-23 00-09-21.hand, na, Yes, 76, 82
gesture6Right, dataOutput/test5/2017-08-23 00-23-59.hand, na, Yes, 83, 89
gesture6Right, dataOutput/test6/2017-08-24 02-34-39.hand, na, Yes, 86, 90

gesture7Right
gesture7Right, dataOutput/General/2017-08-10 01-55-44.hand, c7, Yes, 86, 88
gesture7Right, dataOutput/Alex/2017-08-18 12-56-54.hand, na, Yes, 84, 90
gesture7Right, dataOutput/Alex/2017-08-18 12-57-40.hand, na, Yes, 86, 86
gesture7Right, dataOutput/Jacqueline/2017-08-10 16-03-27.hand, na, Yes, 81, 84
gesture7Right, dataOutput/Stefan/2017-08-23 17-37-58.hand, na, Yes, 54, 68
gesture7Right, dataOutput/test1/2017-08-22 22-38-25.hand, 1, Yes, 76, 65
gesture7Right, dataOutput/test2/2017-08-22 22-44-01.hand, 1, Yes, 85, 75
gesture7Right, dataOutput/test3/2017-08-22 23-36-26.hand, 1, Yes, 83, 80
gesture7Right, dataOutput/test4/2017-08-23 00-09-31.hand, na, Yes, 87, 81
gesture7Right, dataOutput/test5/2017-08-23 00-24-22.hand, na, Yes, 87, 86
gesture7Right, dataOutput/test6/2017-08-24 02-35-28.hand, na, Yes, 87, 84

gesture8Right
gesture8Right, dataOutput/General/2017-08-10 01-56-06.hand, c8, Yes, 84, 77
gesture8Right, dataOutput/Alex/2017-08-18 12-56-58.hand, na, Yes, 80, 69
gesture8Right, dataOutput/Alex/2017-08-18 12-57-43.hand, na, Yes, 87, 83
gesture8Right, dataOutput/Jacqueline/2017-08-10 16-03-46.hand, na, Yes, 75, 67
gesture8Right, dataOutput/Stefan/2017-08-23 17-38-03.hand, na, Yes, 85, 81
gesture8Right, dataOutput/test1/2017-08-22 22-38-39.hand, 1, Yes, 96, 84
gesture8Right, dataOutput/test2/2017-08-22 22-44-19.hand, 1, Yes, 97, 89
gesture8Right, dataOutput/test3/2017-08-22 23-36-48.hand, 1, Yes, 8, 70
gesture8Right, dataOutput/test4/2017-08-23 00-09-41.hand, na, Yes, 94, 85
gesture8Right, dataOutput/test5/2017-08-23 00-24-54.hand, na, Yes, 73, 82
gesture8Right, dataOutput/test6/2017-08-24 02-35-52.hand, na, Yes, 95, 90

gesture9Right
gesture9Right, dataOutput/General/2017-08-10 01-56-25.hand, c9, Yes, 93, 84
gesture9Right, dataOutput/Alex/2017-08-18 12-57-02.hand, na, Yes, 81, 87
gesture9Right, dataOutput/Alex/2017-08-18 12-57-46.hand, na, Yes, 79, 79
gesture9Right, dataOutput/Jacqueline/2017-08-10 16-05-05.hand, na, Yes, 92, 90
gesture9Right, dataOutput/Stefan/2017-08-23 17-38-07.hand, na, Yes, 82, 80
gesture9Right, dataOutput/test1/2017-08-22 22-38-56.hand, 1, Yes, 88, 79
gesture9Right, dataOutput/test2/2017-08-22 22-44-49.hand, 1, Yes, 72, 69
gesture9Right, dataOutput/test3/2017-08-22 23-37-18.hand, 1, Yes, 93, 81
gesture9Right, dataOutput/test4/2017-08-23 00-10-19.hand, na, Yes, 74, 72
gesture9Right, dataOutput/test5/2017-08-23 00-25-30.hand, na, Yes, 90, 85
gesture9Right, dataOutput/test6/2017-08-24 02-37-52.hand, na, Yes, 84, 84

gesture10Right
gesture10Right, dataOutput/General/2017-08-10 01-56-34.hand, c10, Yes, 69, 74
gesture10Right, dataOutput/Alex/2017-08-18 12-57-05.hand, na, Yes, 79, 82
gesture10Right, dataOutput/Alex/2017-08-18 12-57-49.hand, na, Yes, 83, 80
gesture10Right, dataOutput/Jacqueline/2017-08-10 16-05-36.hand, na, Yes, 73, 78
gesture10Right, dataOutput/Stefan/2017-08-23 17-38-16.hand, na, Yes, 84, 79
gesture10Right, dataOutput/test1/2017-08-22 22-39-20.hand, 1, Yes, 81, 85
gesture10Right, dataOutput/test2/2017-08-22 22-45-08.hand, 1, Yes, 84, 82
gesture10Right, dataOutput/test3/2017-08-22 23-38-31.hand, 1, Yes, 91, 89
gesture10Right, dataOutput/test4/2017-08-23 00-10-30.hand, na, Yes, 80, 79
gesture10Right, dataOutput/test5/2017-08-23 00-25-48.hand, na, Yes, 86, 77
gesture10Right, dataOutput/test6/2017-08-24 02-36-47.hand, na, Yes, 91, 82


gesture8Left size: 11 , 11 angleAvg: 82 componentAvg: 81
gesture9Right size: 11 , 11 angleAvg: 84 componentAvg: 80
gesture3Left size: 11 , 11 angleAvg: 83 componentAvg: 83
gesture6Right size: 11 , 11 angleAvg: 68 componentAvg: 81
gesture2Right size: 11 , 11 angleAvg: 58 componentAvg: 78
gesture5Right size: 11 , 11 angleAvg: 89 componentAvg: 81
gesture4Left size: 11 , 11 angleAvg: 90 componentAvg: 91
gesture7Right size: 11 , 11 angleAvg: 81 componentAvg: 80
gesture6Left size: 11 , 11 angleAvg: 85 componentAvg: 80
gesture1Right size: 11 , 11 angleAvg: 73 componentAvg: 86
gesture2Left size: 11 , 11 angleAvg: 85 componentAvg: 79
gesture9Left size: 11 , 11 angleAvg: 89 componentAvg: 81
gesture4Right size: 11 , 11 angleAvg: 76 componentAvg: 88
gesture10Left size: 11 , 11 angleAvg: 78 componentAvg: 76
gesture8Right size: 11 , 11 angleAvg: 79 componentAvg: 79
gesture1Left size: 11 , 11 angleAvg: 83 componentAvg: 87
gesture7Left size: 11 , 11 angleAvg: 29 componentAvg: 77
gesture3Right size: 11 , 11 angleAvg: 87 componentAvg: 82
gesture5Left size: 11 , 11 angleAvg: 51 componentAvg: 87
gesture10Right size: 11 , 11 angleAvg: 81 componentAvg: 80

Averages for each gestureType:
gesture1Right angleAvg: 73 componentAvg: 86
gesture2Right angleAvg: 58 componentAvg: 78
gesture3Right angleAvg: 87 componentAvg: 82
gesture4Right angleAvg: 76 componentAvg: 88
gesture5Right angleAvg: 89 componentAvg: 81
gesture6Right angleAvg: 68 componentAvg: 81
gesture7Right angleAvg: 81 componentAvg: 80
gesture8Right angleAvg: 79 componentAvg: 79
gesture9Right angleAvg: 84 componentAvg: 80
gestue10Right angleAvg: 81 componentAvg: 80

gesture1Left angleAvg: 83 componentAvg: 87
gesture2Left angleAvg: 85 componentAvg: 79
gesture3Left angleAvg: 83 componentAvg: 83
gesture4Left angleAvg: 90 componentAvg: 91
gesture5Left angleAvg: 51 componentAvg: 87
gesture6Left angleAvg: 85 componentAvg: 80
gesture7Left angleAvg: 29 componentAvg: 77
gesture8Left angleAvg: 82 componentAvg: 81
gesture9Left angleAvg: 89 componentAvg: 81
gestue10Left angleAvg: 78 componentAvg: 76


%----------------------------------------------------------------------------------------
%	BIBLIOGRAPHY
%----------------------------------------------------------------------------------------

\printbibliography[heading=bibintoc]

%----------------------------------------------------------------------------------------

\end{document}  
